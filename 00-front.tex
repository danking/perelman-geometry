% !TEX root = perelman-geometry.tex
%!TEX TS-program = pdflatex
%!TEX encoding = UTF-8 Unicode


\maketitle
\cleardoublepage
\thispagestyle{empty}
%\pagenumbering{roman}
\begin{center}

{\LARGE Ya. I. Perelman}

{\Huge Geometry for Entertainment}




The Mir Titles Project

\end{center}
\cleardoublepage

\thispagestyle{empty}
\vfill

{\noindent
Seventh Edition, Revised

Edited and supplemented by B. A. Kordemsky

First Published by State Publishing House Of Technical And Theoretical Literature Moscow -- 1950 -- Leningrad.

Original scan in Russian by the Russian Lutherean on The Internet Archive \url{https://archive.org/details/20220910_perelman_geometry/}.

Translated from the Russian and typeset in \LaTeX{} by \emph{Damitr Mazanav}.

This fully electronic English translation released on the web by \emph{The Mir Titles Project}. \url{https://mirtitles.org} in 2024.

 


Licence Creative Commons by SA 4.0

\cleardoublepage

 \tableofcontents
 
 \cleardoublepage

\chapter{Editor's Preface}
\label{editor-preface}
%\addcontentsline{toc}{chapter}{\nameref{preface}}


\emph{Geometry for Entertainment} is written both for friends of mathematics and for those readers from whom many attractive aspects of mathematics have somehow been hidden.

More importantly, this book is intended for those readers who studied (or are currently studying) geometry only at the blackboard and therefore are not used to noticing familiar geometric relationships in the world of things and phenomena around us, have not learnt to use the acquired geometric knowledge in practise, in difficult cases of life, on a hike, in a bivouac or front-line situation.

To arouse the reader's interest in geometry or, in the words of the author, ``to inspire a desire and cultivate a taste for its study is the objective of this book.''

To this end, the author will take geometry ``out of the walls of the school room into the free air, into the forest, field, to the river, on the road, in order to indulge in relaxed geometric studies without a textbook and tables in the open air \ldots{}'', and draws the reader's attention to the pages of L. N. Tolstoy and A. P. Chekhov, Jules Verne and Mark Twain. He finds a theme for geometric problems in the works of N. V. Gogol and A. S. Pushkin, and finally offers the reader ``a motley selection of problems, curious in plot, unexpected in result.''

The seventh edition of \emph{Geometry for Entertainment} is published without the direct participation of the author. Ya. I. Perelman died in Leningrad in 1942.

The new edition of the book contains almost all the articles of the previous edition, newly illustrated, edited and supplemented with facts and information from our Soviet reality, as well as a considerable number (about 30) additional articles.

I was guided by the desire to increase the ``utility coefficient'' of Ya. Perelman's book, to make it even more effective and interesting, involving new readers in the ranks of friends of mathematics.

To what extent this was possible, I hope to learn from readers at the address: Moscow, 64, Chernyshevsky Str., 81, Sq. 53, B. A. Kordemsky.


\begin{flushright}
\emph{B. Kordemsky}
\end{flushright}


\chapter{Translator's Preface}
\label{translator-preface}
%\addcontentsline{toc}{chapter}{\nameref{preface}}

Yakov Perelman's books have been a constant source of inspiration for me throughout my life. It brings me great pleasure to present this as of now untranslated work of Perelman to English world. 

I have tried my level best to create a readable version of the translation. If there are any mistakes they are all mine. Any suggestions and criticisms to improve the translation are welcome. I hope that this English version finds enthusiastic readers.

\begin{flushright}
\emph{Damitr Mazanav}
\end{flushright}
